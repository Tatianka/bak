\chapter*{Používané pojmy a skratky}
\label{chap:pojmy}

\begin{description}
\item[nsn] -- najmenší spoločný násobok
\item[$ L_{1}L_{2} $] -- zreťazenie jazykov $ L_{1} $ a $ L_{2} $
\item[$ L^* $] -- Kleeneho iterácia ($L^*=\cup^{\infty}_{i=0}L^i$, kde $L^0=\lbrace \varepsilon \rbrace$, $L^1=L$ a $L^{i+1}=L^iL$)
\item[$ \R $] -- trieda regulárnych jazykov, zároveň trieda jazykov tvorená regexami
\item[$ \L_{CF}$] -- trieda bezkontextových jazykov
\item[$ \L_{CS}$] -- trieda bezkontextových jazykov
\item[DKA/NKA] -- deterministický/nedeterministický konečný automat
\item[LBA] -- lineárne ohraničený Turingov stroj
\item[TS] -- Turingov stroj
\item[matchovať] -- keď regex matchuje slovo/vyhlási zhodu, znamená to, že patrí do jeho jazyka
\item[regex] -- regulárny výraz, ktorý môže vytvoriť najviac regulárny jazyk (základná definícia)
\item[e-regex] -- regex so spätnými referenciami
\item[le-regex] -- e-regex s operáciami lookahead a lookbehind
\item[nle-regex] -- le-regex s operáciami negatívny lookahead a negatívny lookbehind
\item[Eregex] -- trieda jazykov tvorená e-regexami
\item[LEregex] -- trieda jazykov tvorená le-regexami
\item[nLEregex] -- trieda jazykov tvorená nle-regexami
\item[lookaround] -- spoločný názov pre lookahead a lookbehind
\end{description}

