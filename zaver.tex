%\cleardoublepage
\chapter*{Záver}\label{chap:conc}
\markboth{Záver}{}
\phantomsection
\addcontentsline{toc}{chapter}{Záver}

Ukázali sme, že regulárne výrazy s mnohými novými konštrukciami stále pokrývajú triedu regulárnych jazykov. Patria medzi ne aj {\ttfamily \selectfont ??,+?,$*$?} a lookaround. Z toho vyplýva, že regexy využívajúce operácie z pomerne veľkej množiny vieme vykonávať algoritmom s lineárnou časovou zložitosťou.

Zistili sme, že regulárne a kontextové jazyky sú uzavreté na lookaround, ale bezkontextové nie sú. Čo sa týka regexov s touto operáciou navyše, trieda jazykov sa nezmenila. Museli sme sa však popasovať s postavením lookaroundu vnútri Kleeneho $*$, pretože každá iterácia jeden lookaround pridala. Ako vidíme, aj toto konečné automaty zvládli.

Zaujímavejšie to začalo byť pri modeli so spätnými referenciami, ktorý sám dokáže popísať kontextový jazyk. Lookahead a lookbehind ho posunuli ešte ďalej. Sprostredkovali uzavretosť na prienik a pomohli popísať jazyky, ktoré nie je možné pumpovať a teda začínajú byť sami o sebe zložité -- napríklad jazyk všetkých platných výpočtov Turingovho stroja. O unárnych jazykoch sa nám pumpovaciu lemu takmer podarilo ukázať, lenže situáciu opäť skomplikovala kombinácia lookaroundu a Kleeneho $*$. Pre tento prípad sme pumpovanie nájsť nedokázali a problém zostal otvorený.

Negatívna verzia lookaheadu a lookbehindu pridala uzavretosť na komplement.

Keďže v našom arzenáli chýbala pumpovacia lema, dokázať o nejakom jazyku, že do triedy nepatrí, sa stalo náročnou záležitosťou. Myslíme si však, že trieda nLEregex stále nepokrýva celé kontextové jazyky a problémom by mohol byť napríklad jazyk $a^nb^n$, nakoľko stále nemáme k dispozícii počítadlo alebo niečo tomu podobné. Otázkou je tiež vymazávajúci homomorfizmus, ktorý by mohol ovplyvniť lookaround. Keby sme napríklad vymazali mriežky z jazyka platných výpočtov TS, lookaround nevie ako ďaleko sa má pozerať -- mriežky ho v podstate navigovali. Táto vlastnosť podobne ovplyňuje aj zreťazenie a iteráciu. Je otázne, či vieme takúto navigáciu niečím nahradiť alebo strata významných znakov vedie k jazyku mimo triedy.

Inšpiráciou pre pokračovanie nášho výskumu by mohli byť vyššie uvedené otvorené problémy. Spätné referencie a lookaround dokopy pôsobia zložito, preto by sa dalo zamyslieť nad vhodným modelom pre ľahšie uchopenie tohto arzenálu. Takisto by mohlo byť zaujímavé nájsť operáciu, ktorá rozšíri triedu nLEregex ďalej, za hranicu kontextových jazykov alebo sa snažiť dosiahnuť úplné pokrytie tried Chomského hierarchie, či už kontextové respektíve bezkontextové jazyky.
