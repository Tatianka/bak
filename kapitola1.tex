\chapter{Názov kapitoly 1}
\label{chap:kapitola1}

V tejto kapitole formálne definujem operácie z uvedenej dokumentácie jazyka Python \cite{Python3Documentation} a ukážem ich silu. Budem používať nasledovné zápisy:

$ L_{1}L_{2} $ -- zreťazenie jazykov $ L_{1} $ a $ L_{2} $

$ L^* $ -- iterácia ($L^*=\cup^{\infty}_{i=0}L^i$, kde $L^0=\lbrace \varepsilon \rbrace$, $L^1=L$ a $L^{i+1}=L^iL$)

$ \R $ -- tradičné označenie triedy regulárnych jazykov

DKA/NKA -- deterministický/nedeterministický konečný automat

\section{Podnadpis 1}
\label{chap:podkapitola1}

\begin{df}[Greedy iterácia]
$$ L_{1} \circledast L_{2} = \lbrace uv ~|~ u \in L_1^* \land v \in L_2 \land u~je~najdlhšie~také \rbrace$$
\end{df}

\begin{df}[Minimalistická iterácia]
$$ L_{1} *? L_{2} = \lbrace uv ~|~ u \in L_1^* \land v \in L_2 \land u~je~najkratšie~také \rbrace $$
\end{df}

\begin{veta}
$L_1 \circledast L_2 = L_1 *? L_2 = L_1^*L_2$
\end{veta}
\begin{proof}
$\subseteq:$
Nech $w \in L_1 \circledast L_2$. Potom z definície $w=uv$ vieme, že $u \in L_1^*$ a $v \in L_2$, teda $uv \in L_1^*L_2$. Analogicky ak $x=yz \in L_1 *? L_2$, potom $yz \in L_1^*L_2$.

$\supseteq:$
Majme $w \in L_1^*L_2$ a rozdeľme na podslová $u,v$ tak, že $u \in L_1^*, v \in L_2$ a $w=uv$. Takéto rozdelenie musí byť aspoň jedno. Ak je ich viac, vezmime to, kde je $u$ najdlhšie. Potom $uv \in L_1 \circledast L_2$. Ak zvolíme $u$ najkratšie, tak zasa $uv \in L_1 *? L_2$.
\end{proof}

\begin{dosledok}
Trieda $\R$ je uzavretá na operácie $\circledast$ a $*?$.
\end{dosledok}

\begin{df}[Lookahead]
$$ L_{1}(?=L_{2})L_{3} = \lbrace uvw ~|~ u \in L_{1} \land v \in L_{2} \land vw \in L_{3} \rbrace $$ Operáciu $(?=\dots)$ nazývame lookahead.
\end{df}

\begin{veta}
Nech $ L_{1}, L_{2}, L_{3} \in \R $. Potom $ L = L_{1}(?=L_{2})L_{3} \in \R $.
\end{veta}
\begin{proof}
Nech $ L_{1},L_{2},L_{3} $ sú regulárne, nech $ A_{i} = (K_{i},\Sigma_{i},\delta_{i},q_{0i},F_{i}) $ sú DKA také, že $ L(A_{i})=L_{i}, i \in \lbrace 1,2,3\rbrace $. Zostrojím NKA $ A = (K,\Sigma,\delta,q_{0},F) $ pre $L$, kde
$ K = K_{1} \cup K_{2} \times K_{3} \cup K_{3} ~ ( $predp. $ K_{1} \cap K_{3}= \emptyset), ~
\Sigma=\Sigma_{1}\cup\Sigma_{2}\cup\Sigma_{3}, ~ q_{0}=q_{01}, ~ F = F_{3} \cup F_{2} \times F_{3}, ~ \delta $~funkciu definujeme nasledovne:
\begin{eqnarray*}
\forall q \in K_{1}, \forall a \in \Sigma &:& \delta(q,a) \ni \delta_{1}(q,a) \\
\forall q \in F_{1} &:& \delta(q,\varepsilon ) \ni \left[ q_{02},q_{03} \right] \\
\forall p \in K_{2}, \forall q \in K_{3}, \forall a \in \Sigma_{2} \cap \Sigma_{3} &:& \delta( \left[ p,q \right] ,a) \ni \left[ \delta(p,a), \delta (q,a) \right] \\
\forall p \in F_2, \forall q \in K_3 &:& \delta(\left[p,q\right],a) \ni \delta(q,a) 
\end{eqnarray*}

$ L(A) = L. $

$ \supseteq: $ Máme $ w \in L $ a chceme preň nájsť výpočet na $A$. Z definície $L$ vyplýva $w=xyz$, kde $x \in L_1, y \in L_2$ a $yz \in L_3$, teda existujú akceptačné výpočty pre $x,y,yz$ na DKA $A_1,A_2,A_3$. Z toho vyskladáme výpočet pre $w$ na $A$ nasledovne. Výpočet pre $x$ bude rovnaký ako na $A_1$. Z akceptačné stavu $A_1$ vieme na $\varepsilon$ prejsť do stavu $\left[q_{02},q_{03}\right]$, kde začne výpočet pre $y$. Ten vyskladáme z $A_2$ a $A_3$ tak, že si ich výpočty napíšeme pod seba a stavy nad sebou budú tvoriť karteziánsky súčin stavov v $A$ (keďže $A_2$ aj $A_3$ sú deterministické, tieto výpočty na $y$ budú rovnako dlhé). $y \in L_2$, teda $A_2$ skončí v akceptačnom stave. Podľa $\delta$ funkcie v $A$ vieme pokračovať len vo výpočte na $A_3$, teda doplníme zvyšnú postupnosť stavov pre výpočet $z$. Keďže $yz \in L_3$ a $F_3\subseteq F$ (resp. $F_2\times F_3\subseteq F$ pre $z=\varepsilon$), automat A akceptuje. 

$ \subseteq: $ Nech $w \in L(A)$, potom existuje akceptačný výpočet na $A$. Z toho vieme $w$ rozdeliť na $x,y$ a $z$ tak, že $x$ je slovo spracovávené od začiatku po prvý príchod do stavu $\left[q_{02},q_{03}\right]$, $y$ odtiaľto po posledný stav reprezentovaný karteziánskym súčinom stavov a zvyšok bude $z$. Nevynechali sme žiadne znaky a nezmenili poradie, teda $w=xyz$. Do $\left[q_{02},q_{03}\right]$ sa $A$ môže prvýkrát dostať len vtedy, ak bol v akceptačnom stave $A_1$. Prechod do $\left[q_{02},q_{03}\right]$ je na $\varepsilon$, takže $x \in L_1$. Práve tento stav je počiatočný pre $A_2$ aj $A_3$. Ak $z=\varepsilon$, tak akceptačný stav $A$ je z $F_2\times F_3$ a $y \in L_2, y \in L_3$ a aj $yz \in L_3$. Z toho podľa definície vyplýva, že $xyz=w \in L$. Ak $z\neq \varepsilon$, potom je akceptačný stav $A$ z $F_3$. Podľa $\delta$ funkcie sa z karteziánskeho súčinu stavov do normálneho stavu dá prejsť len tak, že $A_2$ akceptuje, teda $y \in L_2$. $A_3$ akceptuje na konci, čo znamená $yz \in L_3$. Znova podľa definície operácie lookahead $xyz=w \in L$.
\end{proof}

\begin{df}[Lookbehind]
$$ L_{1}(?<=L_{2})L_{3} = \lbrace uvw ~|~ uv \in L_{1} \land v \in L_{2} \land w \in L_{3} \rbrace $$ Operáciu $(?<=\dots)$ nazývame lookbehind.
\end{df}

\begin{veta}
Nech $ L_{1}, L_{2}, L_{3} \in \R $. Potom $ L = L_{1}(?<=L_{2})L_{3} \in \R $.
\end{veta}
\begin{proof}
Podobne ako pri lookahead. (Karteziánsky súčin stavov $L_1$ a $L_2$, ale $A_2$ sa pripája v každom stave $A_1$ - celkový NKA si potom nedeterministicky zvolí jeden moment tohto napojenia.)
\end{proof}


\todo lookbehind - podobne ako lookahead

\begin{veta}
$ \L_{CF} $ nie je uzavretá na operácie lookahead a lookbehind.
\end{veta}
\begin{proof}
Nech $ L_{1}, L_{2}, L_{3}, L_4 \in \L_{CF} $. $ L_1 = \lbrace a^nb^n ~|~ n\geq 1 \rbrace , L_2 = \lbrace a*b^nc^n ~|~ n\geq 1\rbrace , L_3 = \lbrace a^nb^nc* ~|~ n \geq 1\rbrace, L_4 = \lbrace ab^nc^n ~|~ n \geq 1 \rbrace$. Potom $ d(?=L_1)L_2 = \lbrace da^nb^nc^n ~|~ n\geq 1 \rbrace $ a $ L_3(?<=L_4)d = \lbrace a^nb^nc^nd ~|~ n\geq 1 \rbrace$, čo nie sú bezkontextové jazyky.
\end{proof}

\begin{veta}
$ \L_{CS} $ je uzavretá na operáciu lookahead.
\end{veta}
\begin{proof}
Pre $ L_{1}, L_{2}, L_{3} \in \L_{CS} $ a slovo z $ L_{1}(?<=L_{2})L_{3} $ simulujeme najprv LBA pre $L_1$ a potom na 2 stopách súčasne LBA pre $L_2$ a $L_3$. Ak skončia súčasne, potom akceptujeme ak akceptujú oba. Ak LBA pre $L_2$ skončí skôr a akceptuje, tak aj my akceptujeme. V ostatných prípadoch sa zasekneme/zacyklíme - neakceptujeme.
\end{proof}

\section{Podnadpis 2}\label{chap:podkapitola2}

