\chapter{Naše výsledky}
\label{chap:vysledky}

\section{Minimalistická iterácia}
\label{iteracia}

\begin{veta}
$L_1 \circledast L_2 = L_1 *? L_2 = L_1^*L_2$
\end{veta}
\begin{proof}
$\subseteq:$
Nech $w \in L_1 \circledast L_2$. Potom z definície $w=uv$ vieme, že $u \in L_1^*$ a $v \in L_2$, teda $uv \in L_1^*L_2$. Analogicky ak $x=yz \in L_1 *? L_2$, potom $yz \in L_1^*L_2$.

$\supseteq:$
Majme $w \in L_1^*L_2$ a rozdeľme na podslová $u,v$ tak, že $u \in L_1^*, v \in L_2$ a $w=uv$. Takéto rozdelenie musí byť aspoň jedno. Ak je ich viac, vezmime to, kde je $u$ najdlhšie. Potom $uv \in L_1 \circledast L_2$. Ak zvolíme $u$ najkratšie, tak zasa $uv \in L_1 *? L_2$.
\end{proof}

\begin{dosledok}
Trieda $\R$ je uzavretá na operácie $\circledast$ a $*?$.
\end{dosledok}

Kleeneho $*$ uvedená v definícii regexov je príliš nedeterministická na ľahké prevedenie do praxe, preto reálny model používa v prípade operácie $*$ algoritmus pre greedy iteráciu. Vidíme však, že to nevadí, lebo pokrývame stále tú istú triedu jazykov. Takisto po pridaní minimalistickej iterácie. Z teoretického hľadiska je existencia dvoch operácií s rovnakou funkcionalitou  zbytočná. Ak zhoda existuje, regex matchuje s použitím ľubovoľnej z nich. Ale riešenie (rozdelenie slova) vyzerá inak, čo má v praxi zmysel pri jeho ďalšom použití. Preto je existencia oboch operácií opodstatnená.

\section{Lookaround}
\label{chap:lookahead}

\subsection{Chomského hierarchia}

Operácie máme zadefinovné, poďme sa pozrieť, čo urobia s jazykmi z tried Chomského hierarchie.

\begin{lema}
Trieda $\R$ je uzavretá na lookahead.
\end{lema}
\begin{proof}
Nech $ L_{1}, L_{2}, L_{3} \in \R $, chceme ukázať, že  $ L = L_{1}(?=L_{2})L_{3} \in \R $.

Keďže $ L_{1},L_{2},L_{3} $ sú regulárne, existujú DKA $ A_{i} = (K_{i},\Sigma_{i},\delta_{i},q_{0i},F_{i}) $ také, že $ L(A_{i})=L_{i}, i \in \lbrace 1,2,3\rbrace $. 

Zostrojíme NKA $ A $ pre $L$. Myšlienka spočíva v tom, že najprv necháme simulovať $A_1$ a keď akceptuje, simulujeme naraz $A_2$ a $A_3$. Konštrukcia je podoná ako prienik dvoch regulárnych jazykov (karteziánsky súčin stavov) s tým rozdielom, že $A_2$ môže skončiť skôr ako $A_3$.

\textbf{Konštrukcia.}  $ A = (K,\Sigma,\delta,q_{0},F) $, kde
$ K = K_{1} \cup K_{2} \times K_{3} \cup K_{3} ~ ( $predp. $ K_{1} \cap K_{3}= \emptyset), ~
\Sigma=\Sigma_{1}\cup\Sigma_{2}\cup\Sigma_{3}, ~ q_{0}=q_{01}, ~ F = F_{3} \cup F_{2} \times F_{3}, ~ \delta $~funkciu definujeme nasledovne:
\begin{eqnarray*}
\forall q \in K_{1}, \forall a \in \Sigma &:& \delta(q,a) \ni \delta_{1}(q,a) \\
\forall q \in F_{1} &:& \delta(q,\varepsilon ) \ni \left[ q_{02},q_{03} \right] \\
\forall p \in K_{2}, \forall q \in K_{3}, \forall a \in \Sigma_{2} \cap \Sigma_{3} &:& \delta( \left[ p,q \right] ,a) \ni \left[ \delta(p,a), \delta (q,a) \right] \\
\forall p \in F_2, \forall q \in K_3 &:& \delta(\left[p,q\right],a) \ni \delta(q,a) 
\end{eqnarray*}

$ L(A) = L. $

$ \supseteq: $ Máme $ w \in L $ a chceme preň nájsť výpočet na $A$. Z definície $L$ vyplýva $w=xyz$, kde $x \in L_1, y \in L_2$ a $yz \in L_3$, teda existujú akceptačné výpočty pre $x,y,yz$ na DKA $A_1,A_2,A_3$. Z toho vyskladáme výpočet pre $w$ na $A$ nasledovne. Výpočet pre $x$ bude rovnaký ako na $A_1$. Z akceptačné stavu $A_1$ vieme na $\varepsilon$ prejsť do stavu $\left[q_{02},q_{03}\right]$, kde začne výpočet pre $y$. Ten vyskladáme z $A_2$ a $A_3$ tak, že si ich výpočty napíšeme pod seba a stavy nad sebou budú tvoriť karteziánsky súčin stavov v $A$ (keďže $A_2$ aj $A_3$ sú deterministické, tieto výpočty na $y$ budú rovnako dlhé). $y \in L_2$, teda $A_2$ skončí v akceptačnom stave. Podľa $\delta$--funkcie v $A$ vieme pokračovať len vo výpočte na $A_3$, teda doplníme zvyšnú postupnosť stavov pre výpočet $z$. Keďže $yz \in L_3$ a $F_3\subseteq F$ (resp. $F_2\times F_3\subseteq F$ pre $z=\varepsilon$), automat A akceptuje. 

$ \subseteq: $ Nech $w \in L(A)$, potom existuje akceptačný výpočet na $A$. Z toho vieme $w$ rozdeliť na $x,y$ a $z$ tak, že $x$ je slovo spracovávené od začiatku po prvý príchod do stavu $\left[q_{02},q_{03}\right]$, $y$ odtiaľto po posledný stav reprezentovaný karteziánskym súčinom stavov a zvyšok bude $z$. Nevynechali sme žiadne znaky a nezmenili poradie, teda $w=xyz$. Do $\left[q_{02},q_{03}\right]$ sa $A$ môže prvýkrát dostať len vtedy, ak bol v akceptačnom stave $A_1$. Prechod do $\left[q_{02},q_{03}\right]$ je na $\varepsilon$, takže $x \in L_1$. Práve tento stav je počiatočný pre $A_2$ aj $A_3$. Ak $z=\varepsilon$, tak akceptačný stav $A$ je z $F_2\times F_3$ a $y \in L_2, y \in L_3$ a aj $yz \in L_3$. Z toho podľa definície vyplýva, že $xyz=w \in L$. Ak $z\neq \varepsilon$, potom je akceptačný stav $A$ z $F_3$. Podľa $\delta$--funkcie sa z karteziánskeho súčinu stavov do normálneho stavu dá prejsť len tak, že $A_2$ akceptuje, teda $y \in L_2$. $A_3$ akceptuje na konci, čo znamená $yz \in L_3$. Znova podľa definície operácie lookahead $xyz=w \in L$.
\end{proof}

\begin{lema}\label{lb+R}
Trieda $\R$ je uzavretá na lookbehind.
\end{lema}
\begin{proof}
Opäť chceme ukázať, že $ L = L_{1}(?<=L_{2})L_{3} \in \R $ pre $ L_{1}, L_{2}, L_{3} \in \R $. Postupujeme podobne ako pri lookahead. Urobíme karteziánsky súčin stavov $A_1$ a $A_2$ tak, že $A_2$ môže začať výpočet v ľubovoľnom stave $A_1$ - celkový NKA si potom nedeterministicky zvolí jeden moment tohto napojenia. $A_1$ a $A_2$ musia naraz akceptovať, potom sa začne simulácia $A_3$.
\end{proof}

\begin{veta}\label{lookahead+R}
Regulárne jazyky sú uzavreté na lookaround.
\end{veta}

\begin{veta}
$ \L_{CF} $ nie je uzavretá na operácie lookahead a lookbehind.
\end{veta}
\begin{proof}
Nech $ L_{1}, L_{2}, L_{3}, L_4 \in \L_{CF} $. $ L_1 = \lbrace a^nb^n ~|~ n\geq 1 \rbrace , L_2 = \lbrace a*b^nc^n ~|~ n\geq 1\rbrace , L_3 = \lbrace a^nb^nc* ~|~ n \geq 1\rbrace, L_4 = \lbrace ab^nc^n ~|~ n \geq 1 \rbrace$. Potom $ d(?=L_1)L_2 = \lbrace da^nb^nc^n ~|~ n\geq 1 \rbrace $ a $ L_3(?<=L_4)d = \lbrace a^nb^nc^nd ~|~ n\geq 1 \rbrace$, čo nie sú bezkontextové jazyky.
\end{proof}

\begin{veta} \label{CS-lookahead}
$ \L_{CS} $ je uzavretá na operáciu lookahead a lookbehind.
\end{veta} 
\begin{proof}
Uzavretosť na lookahead:

Nech $ L_{1}, L_{2}, L_{3} \in \L_{CS} $. Pre $ L = L_{1}(?=L_{2})L_{3} $ zostrojíme LBA $A$ z LBA $A_1, A_2, A_3$ pre dané kontextové jazyky. Najprv sa pozrime na štruktúru vstupu -- prvé je slovo z $L_1$ a za ním nasleduje slovo z $L_3$, pričom jeho prefix patrí do $L_2$. Preto, aby $A$ mohol simulovať dané lineárne ohraničené automaty, je potrebné označiť hranice jednotlivých slov.

 Na začiatku výpočtu $A$ prejde pásku a nedeterministicky označí 2 miesta -- koniec slov pre $A_1$ a $A_2$. Následne sa vráti na začiatok a simuluje $A_1$. Ak akceptuje, $A$ pokračuje a presunie sa za označený koniec vstupu pre $A_1$. Inak sa zasekne. V tomto bode sa začína vstup pre $A_2$ aj $A_3$, teda slovo až do konca prepíše na 2 stopy. Najprv na hornej simuluje $A_2$. Pokiaľ $A_2$ neskončí v akceptačnom stave, $A$ sa zasekne. Inak sa vráti na označené miesto a simuluje $A_3$ na spodnej stope až do konca vstupu. Akceptačný stav $A_3$ znamená akceptáciu celého vstupného slova.

Uzavretosť na lookbehind sa dokáže analogicky.
\end{proof}

\subsection{Regexy}

Keď už sme lepšie zoznámení s lookaroundom, mali by sme posúdiť ako zapadá medzi regexy. Nebude to také jednoduché, keďže celý model regexov je množina operácií a jednopísmenkových jazykov (konečná abeceda), z ktorých postupne vyskladávame zložitejšie jazyky. V Chomského hierarchii sme pracovali iba s akousi finálnou formou jazyka, o ktorom sme vedeli zopár vlastností. Teraz zoberieme množinu operácií a pridáme k nim ďalšie dve. Otázkou je, čo spôsobí ich kombinovanie. Najzaujímavejšie vyzerajú tie, ktoré vedia ovplyvniť samotné vlastnosti lookaroundu, nakoľko sme ukázali, že regulárne jazyky sú naň uzavreté.

\begin{lema}\label{lookahead s *}
Nech $L_1,L_2,L_3,L_4 \in \R$, $\alpha = \left( L_1 \left( ?=L_2\right) L_3 \right) * L_4$. Potom $L(\alpha) \in \R$.
\end{lema} 
\begin{proof}
Keďže $L_1,L_2,L_3,L_4 \in \R$, tak pre ne existujú DKA $A_1,A_2,A_3,A_4$, kde $A_i = \left( K_i, \Sigma_i, \delta_i, q_{0i}, F_i \right)$ pre $\forall i$. Z nich zostrojíme NKA $A$ pre $L$. Výpočet bez lookaheadov by vyzeral tak, že by sme simulovali $A_1$, potom po jeho akceptácii $A_3$ a odtiaľ by sa išlo v rámci iterácie naspäť na $A_1$. Zároveň z $A_1$ by sa dalo na $\varepsilon$ prejsť na $A_4$, čo by znamenalo koniec iterácie (ošetruje aj nulovú iteráciu). Pozrime sa na to, ako a kam vsunúť lookahead. Problém je, že pri každej ďalšej iterácii pribúda nový, teda ďalší $A_2$. Vieme ich však simulovať všetky naraz, keď vezmeme do úvahy, že vždy pracujeme nad konečnou abecedou a $K_2$ je konečná. Z toho vyplýva, že aj $\P(K_2)$ je konečná, a teda vieme zostrojiť automat, ktorý si bude simulovať prechody medzi množinami stavov $A_2$.

\textbf{Konštrukcia.} $A = (K,\Sigma ,\delta ,q_0,F) : K = (K_1\cup K_3\cup K_4)\times \P(K_2)$, kde $K_1 \cap K_3 \cap K_4 = \emptyset$ (množiny v stavoch možno reprezentovať napr. 0-1 reťazcom dĺžky $\left\vert{K_2}\right\vert$, kde 1 na $i$--tom mieste symbolizuje, že nejaká inštancia $A_2$ je v $i$--tom stave), $\Sigma = \Sigma_1 \cup \Sigma_2 \cup \Sigma_3 \cup \Sigma_4, ~q_0 = (q_{01},\emptyset ), ~ F = F_4\times \emptyset,~\delta$--funkcia:
 
\begin{itemize}
  \item $\forall q \in K_i~i=1,3,4 ~, ~\forall U \in \P(K_2),\forall a \in \Sigma : \delta ((q_i,U),a) \ni (\delta _i (q_i,a),V),$
  
  kde $\forall q \in U:~\delta_2(q,a) \in V'$ a  $V=V'\setminus F_2$
  \item $\forall q_A \in F_1, \forall U \in \P(K_2) : \delta ((q_A,U),\varepsilon ) \ni (q_{03},U \cup \lbrace q_{02} \rbrace)$
  \item $\forall q_A \in F_3, \forall U \in \P(K_2) : \delta ((q_A,U),\varepsilon ) \ni (q_{01},U)$
  \item $\forall U \in \P(K_2) : \delta ((q_{01},U),\varepsilon ) \ni (q_{04},U)$
\end{itemize}

Automat $A$ akceptuje až keď akceptuje $A_4$. Je zrejmé, že ak v simulácii $A_i$ príde písmenko, ktoré do $\Sigma_i$ nepatrí, automat sa zasekne.

$L(A) = L(\alpha).$

$\subseteq:$ Nech $w \in L(A)$, potom preň existuje akceptačný výpočet na $A$. Podľa stavov vieme určiť počet iterácií, podslová z $L_1,L_3,L_4$ a takisto vznikajúce a akceptujúce výpočty na $A_2$ -- každý takýto výpočet totiž začína s výpočtom na $A_3$ a keďže $A_2$ je de\-ter\-mi\-nis\-tic\-ký, existuje práve jeden výpočet, ktorý musí byť akceptačný. Teda vieme povedať, že $w = x_1y_1x_2y_2 \dots x_ny_nz$, kde $n$ je počet iterácií, $\forall i = 1, 2, \dots, n : ~ x_i \in L_1, y_i \in L_3 $ a $z \in L_4$. Zároveň vieme, že v mieste, kde začína $y_i$ takisto začína podreťazec slova $w$, ktorý patrí do $L_2$. Z toho vidíme ako vyzerá matchovanie regexu $\alpha$.

$\supseteq:$ Majme $v \in L(\alpha)$, teda vieme nájsť zhody podslov $v$ pre všetky $L_i$ (rovnaká dekompozícia slova ako v predošlej inklúzii). Keďže poradie jazykov je rovnaké ako $\varepsilon$--ové prepojenie $A_i$ v $A$ (akceptačný--počiatočný, počiatočný--počiatočný pri $L_4$), vieme správne poprepájať akceptačné výpočty jednotlivých $A_i$ do celkového výpočtu automatu $A$.
\end{proof}

\begin{lema}\label{lookbehind s *}
Nech $L_1,L_2,L_3,L_4 \in \R$, $\alpha = L_4 \left( L_1 \left( ?<=L_2\right) L_3 \right) *$. Potom $L(\alpha) \in \R$.
\end{lema} 
\begin{proof}
Opäť dôkaz pre lookbehind vyzerá podobne ako pre lookahead, $A$ simuluje naj\-prv $A_4$, potom $A_1,A_3,A_1,A_3, \dots$ až po koniec slova. Ale musíme zmeniť simulovanie $A_2$. Stále potrebujeme množinu jeho stavov, keďže každou iteráciou má jeden pribudnúť, ale pridanie počiatočného stavu nie je viazané na konkrétny krok -- ten môže teraz pribudnúť hocikedy (nedeterministicky). To, čo je fixné je akceptačný stav. Ten zase musí prísť zarovno s akceptáciou $A_1$. 

Celý výpočet upravíme tak, že $A_4$ sa bude simulovať ako prvý (keďže sa pozmenil celý jazyk $L$), kedykoľvek môže pribudnúť počiatočný stav $A_2$ a vždy keď akceptuje $A_1$, aspoň jeden $A_2$ musí akceptovať. Aspoň jeden preto, že to nemusí byť práve jeden. V akceptačnom stave sa môžu naraz nachádzať hoci aj 2 inštancie $A_2$, ale v tom prípade nechceme z množiny stavov vymazávať jeho akceptačný stav. Preto tento krok necháme na nedeterminizmus. V prípade, že by sme sa nedeterministicky rozhodli pre viacero výpočtov lookbehindov ako je iterovaní a automat akceptuje, pretože sa všetkým $A_2$ podarilo v správnom kroku skončiť (t.j. viaceré skončili naraz), nevadí nám to. Samotný lookbehind totiž môže matchovať vedieť viaceré podslová končiace v jednom bode, čo sa vie prejaviť práve takto. 

$\delta$--funkcia by vyzerala takto:

\begin{itemize}
  \item $\forall U \in \P(K_2) : \delta ((q_{04},U),\varepsilon ) \ni (q_{01},U)$
  \item $\forall q \in K_i~i=1,3,4 ~, ~\forall U \in \P(K_2),\forall a \in \Sigma : \delta ((q_i,U),a) \ni (\delta _i (q_i,a),V),$
  
  kde $\forall q \in U:~\delta_2(q,a) \in V$
  \item $\forall q_A \in F_1 : \delta ((q_A,U),\varepsilon ) \ni (q_{03},U), (q_{03},U')$
  
  pre všetky také $U \in \P(K_2)$, že $ \exists q_A \in F_2$ a $ q_A \in U ~ ; ~ U'=U \setminus q_A$
  \item $\forall q_A \in F_3, \forall U \in \P(K_2) : \delta ((q_A,U),\varepsilon ) \ni (q_{01},U)$
\end{itemize}
$F = F_3 \times \emptyset$. $L(A) = L(\alpha)$ dokážeme podobne ako v leme \ref{lookahead s *}.
\end{proof}

\begin{veta} \label{reg_uz_la}
Trieda nad regexami s lookaheadom je $\R$.
\end{veta}
\begin{proof}
Samotné regexy pokrývajú triedu regulárnych jazykov a tá je na lookaround uzavretá \ref{lookahead+R}. Keďže pracujeme s množinou operácií, treba overiť, či nejaká ich kombinácia nie je náhodou silnejšia. Ak regex umiestnime do lookaroundu, či pred alebo za neho, vždy to bude regulárny jazyk a celý regex bude tiež definovať regulárny jazyk. Teda nás zaujíma vloženie lookaroundu dovnútra inej operácie. V tomto prípade prichádza do úvahy $*,+$ a $?$, ktoré menia počet lookaroundov a vynútia ich simuláciu na rôznych častiach slova.

$?$ veľa nespraví -- lookaround tam buď bude 1x alebo nebude vôbec. $+$ je prípad $*$ s jedným lookaroundom istým. No a podľa liem \ref{lookahead s *} a \ref{lookbehind s *} vieme, že ani táto kombinácia nestačí na zložitejší jazyk.
\end{proof}

\subsection{Vlastnosti lookaroundu}

Tu ukážeme, že dávať do lookaheadu prefixový jazyk nemá zmysel. Vytvorme čisto jazyk všetkých rôznych prefixov, aké obsahuje. Do lookaheadu stačí vložiť regex pre tento jazyk a celkový matchovaný jazyk to nezmení. Samozrejme, to isté platí aj pre lookbehind a sufixové jazyky.

\begin{veta}\label{bezprefixove}
Nech $L$ je ľubovoľný jazyk a $L_p = L \cup \lbrace uv ~|~ u \in L \rbrace$. Nech $\alpha$ je ľubovoľný regulárny výraz taký, že obsahuje $(?=L_p)$. Potom ak prepíšeme tento lookahead na $(?=L)$ (nazvime to $\alpha '$), bude platiť $L(\alpha ') = L(\alpha )$. Analogicky to platí aj pre lookbehind.
\end{veta}
\begin{proof}
$\subseteq :$ triviálne,  $L \subseteq L_p$.

$\supseteq :$ Majme $w \in L(\alpha)$ a nech $x$ je také podslovo $w$, ktoré sa zhodovalo práve s daným lookaheadom. Potom $x \in L_p$, teda $x=uv$, kde $u \in L$. Ak $v=\varepsilon$, $x \in L$ a máme čo sme chceli. Takže $v\neq \varepsilon$. Ale celá zhoda lookaheadu sa môže zúžiť len na $u$, keďže $u \in L_p$, a bude to platná zhoda s $w$. Čo znamená, že $w \in L(\alpha ')$.
\end{proof}

\begin{dosledok}
Nech $\alpha$ je regulárny výraz, ktorý obsahuje nejaký taký lookahead $(?=L)$ (lookbehind $(?<=L)$), že $\varepsilon \in L$. Nech je $\alpha '$ regulárny výraz bez tohto lookaheadu (lookbehindu). Potom $L(\alpha ') = L(\alpha)$.
\end{dosledok}
\begin{proof}
Uvedomme si, že lookaround nie je fixovaný na dĺžku vstupu - musí sa zhodovať s nejakým podslovom začínajúcim sa (končiacim sa) na konkrétnom mieste. Tým pádom akonáhle si môže regulárny výraz vnútri tejto operácie vybrať $\varepsilon$, bude hlásiť zhodu vždy.
\end{proof}

\subsection{Spätné referencie}
\label{la-backref}

Skúsme teraz zložitejší model. Pridáme k e-regexom pozitívny lookaround. Túto množinu nazývame \textbf{le-regex}, triede jazykov nad le-regexami hovoríme \textbf{LEregex}.

\begin{veta}
$ Eregex \subsetneq LEregex $
\end{veta}
\begin{proof}
$ \subseteq $ vyplýva z definície.

Jazyk $L = \lbrace a^iba^{i+1}ba^k ~|~ k=i(i+1)k' ~ pre ~ nejaké ~ k'>0, i>0 \rbrace$ nepatrí do triedy Eregex \cite[Lemma 2]{ExtendedRegexIntersec}, ale patrí do LEregex:
$$ \alpha = (a*)b(\backslash 1a)b(?=(\backslash 1)* \mathdollar )(\backslash 2)* \mathdollar 
\footnote{Nie je nutné dávať $\mathdollar$ na koniec le-regexu, nakoľko pracujeme v teoretickom prostredí a nehľadáme slovo v texte. Na vstup dostávame vždy len samotné slovo a teda vieme povedať, že na jeho konci bude isto aj koniec riadku ($\mathdollar$). Znak sme uviedli, pre ľahšie pochopenie, budeme uvádzať aj vo všetkých ďalších prípadoch.} $$
\end{proof}

\begin{veta}
$ LEregex \subseteq \L_{CS} $
\end{veta}
\begin{proof}
Vieme, že $ Eregex \in \L_{CS} $ \cite[Theorem 1]{ExtendedRegexPower}, teda ľubovoľný e-regex vieme simulovať pomocou LBA. Ukážeme, že ak pridáme operáciu lookahead/lookbehind, vieme to simulovať tiež.

Nech $\alpha$ je le-regex. Potom $A$ je LBA pre $\alpha$, ktorý ignoruje lookaround (t.j. vyrábame LBA pre e-regex). Teraz vytvoríme LBA $B$ pre e-regex vnútri lookaroundu. Z nich vytvoríme LBA $C$ pre úplný le-regex $\alpha$ tak, že bude simulovať $A$ a keď príde na rad lookaround zaznačí si, v akom stave je $A$ a na ktorom políčku skončil, skopíruje slovo na ďalšiu stopu a na nej simuluje od/do toho miesta $B$ (lokahead/lookbehind). Pokiaľ $B$ akceptoval, vráti sa naspäť k výpočtu na $A$. $C$ akceptuje práve vtedy, keď $A$.

Všimnime si, že samotný look\-a\-round môže obsahovať le-regex, t.j. vnorený lookaround. Tých však môže byť iba konečne veľa, keďže každý le-regex musí mať konečný zápis. Čo znamená, že aj stôp bude konečne veľa a naznačeným postupom si vieme postupne vybudovať LBA, ktorý bude simulovať $\alpha$.
\end{proof}

\begin{veta} \label{prienik}
LEregex je uzavretá na prienik.
\end{veta}
\begin{proof} 
Nech $L_1,L_2 \in LEregex$, potom $L_1 \cap L_2 = \left( ?= L_1 \mathdollar \right) L_2 \mathdollar $.
\end{proof}

\begin{veta}
Jazyk všetkých platných výpočtov Turingovho stroja patrí do LEregex.
\end{veta}
\begin{proof}
Máme Turingov stroj $A = (K,\Sigma ,q_0,\delta ,F)$. Zostrojíme le-regex $\alpha$, ktorý bude matchovať všetky jeho platné výpočty. Vieme, že každý TS pracuje nad konečnou abecedou, má konečne veľa stavov a konečne definovanú $\delta$--funkciu. Keďže to všetko treba zahrnúť do le-regexu, vďaka konečnosti máme potrebné predpoklady na jeho vytvorenie. Stav bude ukazovať pozíciu hlavy a $\#$ bude oddeľovať jednotlivé kon\-fi\-gu\-rá\-cie (bude sa nachádzať aj na začiatku a na konci celého slova/výpočtu).

Prejdime k samotnému výpočtu. Na to, aby po sebe nasledovali len prijateľné kon\-fi\-gu\-rá\-cie využijeme lookahead a spätné referencie. To zabezpečí, aby sa v každom kroku výpočtu menilo len malé okolie stavu (spätné referencie zvyšok okopírujú) a aby sa menilo práve podľa $\delta$--funkcie. Keď máme zaručenú správnu postupnosť konfigurácií, vieme použiť $*$ a matchovať tak ľubovoľne dlhý výpočet. Netreba zabudnúť, že prvá konfigurácia je počiatočná a posledná akceptačná. Toto sú špeciálne prípady, preto ich ošetríme zvlášť, a takisto prípad, kedy je počiatočný stav zároveň aj akceptačným.

\textbf{Konštrukcia.} $\alpha = \beta(\gamma)*\eta$, kde:
\begin{itemize}
\item $\gamma = \gamma_1~|~\gamma_2~|~\gamma_3$
\begin{list}{$\star$}{}
\item $ \displaystyle \gamma_1 = (\mathop(_k.*\mathop)_k x q y \mathop(_{k+1} .* \mathop)_{k+1} \#)(?= \xi \#)$ platí pre $\forall q \in K,~\forall y \in \Sigma$ a kde $ \xi = \xi_1 ~|~ \xi_2 ~|~ \dots ~|~ \xi_n $.
\begin{list}{$\circ$}{Ak}
\item $ (p,z,0) \in \delta(q,y)$, potom $\xi_i = (\backslash kx p z \backslash k+1)$ pre nejaké $i$
\item $ (p,z,1) \in \delta(q,y)$, potom $\xi_i = (\backslash kx z p \backslash k+1)$ pre nejaké $i$
\item $ (p,z,-1) \in \delta(q,y)$, potom $\xi_i = (\backslash kp x z \backslash k+1)$ pre nejaké $i$
\end{list}
\item $ \displaystyle \gamma_2 = (q y \mathop(_{k} .* \mathop)_{k+1} \#)(?= \xi \#)$ platí pre $\forall q \in K,~\forall y \in \Sigma \cup \lbrace B \rbrace$\footnote{$B$ označuje blank -- prázdne políčko na páske. Z definície TS $B \notin \Sigma$.} a kde $ \xi = \xi_1 ~|~ \xi_2 ~|~ \dots ~|~ \xi_n $. Teda hlava ukazuje na začiatok slova. Dodefinovanie $\xi_i$ je podobné.
\begin{list}{$\circ$}{Ak}
\item $ (p,z,0) \in \delta(q,y)$, potom $\xi_i = (p z \backslash k)$ pre nejaké $i$
\item $ (p,z,1) \in \delta(q,y)$, potom $\xi_i = (z p \backslash k)$ pre nejaké $i$
\item $ (p,z,-1) \in \delta(q,y)$, potom $\xi_i = (p B z \backslash k)$ pre nejaké $i$ 
\end{list}
\item $ \displaystyle \gamma_3 = (\mathop(_k.*\mathop)_k x q y \#)(?= \xi \#)$ platí pre $\forall q \in K,~\forall y \in \Sigma \cup \lbrace B \rbrace$ a kde $ \xi = \xi_1 ~|~ \xi_2 ~|~ \dots ~|~ \xi_n $. Teda hlava je na konci slova, opäť je treba ošetriť blankový prípad.
\begin{list}{$\circ$}{Ak}
\item $ (p,z,0) \in \delta(q,y)$, potom $\xi_i = (\backslash kx p z)$ pre nejaké $i$
\item $ (p,z,1) \in \delta(q,y)$, potom $\xi_i = (\backslash kx z p B)$ pre nejaké $i$
\item $ (p,z,-1) \in \delta(q,y)$, potom $\xi_i = (\backslash kp x z)$ pre nejaké $i$
\end{list}
\end{list}
\item $\displaystyle \beta = (\#q_0x\mathop(_k.*\mathop)_k\#)(?=\theta\#) $ je definovaná rovnako ako $\gamma_2$, ale iba pre $q_0$.
\item $\eta = ((.*) q_A (.*)\#)$ platí pre $\forall q_A \in F$
\end{itemize}
Ak je počiatočný stav akceptačným, pridáme na koniec $\alpha$ regex '$| (\#q_0.*\#)$' .
\end{proof}

\paragraph{Poznámka.} Práve sme si ukázali aká silná je kombinácia lookaroundu a spätných referencií. Jazyk platných výpočtov TS je celkom zložitý vďaka závislostiam z $\delta$--funkcie a nedá sa pumpovať. Vidíme to, keď si predstavíme DTS, ktorý akceptuje nekonečný jazyk. Tým, že existujú akceptačné výpočty na týchto slovách, získavame nekonečný jazyk platných výpočtov. Nevieme však nadĺžiť ani slovo -- pokazilo by to výpočet, ani výpočet -- vďaka determinizmu vieme, že ak sa TS zacyklí, tak sa z toho už nedostane.

Tento poznatok nielenže zvýrazňuje rozdiely medzi triedami Eregex a LEregex, ale zároveň nám komplikuje situáciu, pretože dôkaz, že nejaký jazyk nepatrí do triedy LEregex je bez pumpovacej lemy veľmi zložitý. Preto sme sa skúsili pozrieť na unárne jazyky, či by aspoň na nich pumpovanie fungovalo.

Vyslovíme vetu, ktorá je veľmi blízko pumpovacej leme, ale ňou nie je, nakoľko sme nevedeli dokázať pumpovanie pre jeden prípad. A to keď sa lookahead, obsahujúci na konci znak $\mathdollar$ (lookbehind so znakom $\textasciicircum$ na začiatku je analogický prípad), objaví vnútri iterácie. Nenašli sme argument ako zaručiť, že pri pumpovaní vzniká slovo z daného jazyka. Problém tkvie v tom, že pri každej pridanej iterácii sa pridá aj ďalší lookahead a my musíme zaručiť, že matchuje slovo za ním až do konca.

\begin{veta}
Nech $\alpha$ taký je le-regex nad unárnou abecedou $\Sigma = \lbrace a \rbrace$, že neobsahuje lookahead s $\mathdollar$ a lookbehind s $\textasciicircum$ vnútri iterácie.  Existuje konštanta $N$ taká, že ak $w \in L(\alpha)$ a $\vert w \vert > N$, potom existuje dekompozícia $w=xy$ s nasledujúcimi vlastnosťami:
\begin{enumerate}
\item $\vert y \vert \geq 1$
\item $\exists k \in \N,~k\neq 0;~\forall j = 1,2,\ldots: xy^{kj} \in L(\alpha)$
\end{enumerate}
\end{veta}
\begin{proof}
Pokiaľ $\alpha \in Eregex$, tak pre $\alpha$ platí pumpovacia lema \cite[Lemma 1]{ExtendedRegexPower}, t.j. $w = a_0ba_1b\dots a_m$ pre nejaké $m$ a $a_0b^ja_1b^j\dots a_m \in L(\alpha)~\forall j$. My pracujeme nad unárnym jazykom, teda na poradí nezáleží: $x=a_0a_1\dots a_m,~y=b^m, k=1$ a $xy^j \in L(\alpha)~\forall j$.

Zaoberajme sa le-regexami, ktoré obsahujú aspoň jeden lookaround. Ale ešte predtým si povedzme, aké je to dostatočne dlhé slovo. Chceme, aby sme s určitosťou vedeli, či nejaká $*/+$ iteruje aspoň 2x. Zoberieme dĺžku le-regexu $\vert \alpha \vert$. Skreslovať ju môžu 3 ope\-rá\-cie -- spätné referencie, $\lbrace n \rbrace$ a $\lbrace n,m \rbrace$. Nech teda $d$ je súčet dĺžok le-regexov vnútri $l$--tých zátvoriek pre všetky $\backslash l$, ktoré v $\alpha$ sú a sú pripočítané toľkokrát, koľkokrát sa tam jednotlivé $\backslash l$ nachádzajú. Ďalej k $d$ pripočítame za každé $\lbrace n \rbrace$ $n$--krát dĺžku le-regexu, ktorý má kopírovať, a takisto za každú operáciu $\lbrace n,m \rbrace$ pripočítame takýto le-regex $m$--krát. Pre $N = \vert \alpha \vert + d$ vieme povedať, že ak $\vert w \vert > N$, potom jediná operácia, ktorá ho vie predĺžiť do takejto dĺžky môže byť iba Kleeneho $*$, prípadne $+$. $+$ je však v podstate prípad $*$, preto budeme pracovať s touto operáciou.

Máme dostatočne dlhé slovo, rozoberieme si teraz pípady, ktoré môžu nastať. 
\begin{enumerate}
\item Žiadny lookaround nezasahuje do $*$, ktorá iteruje. V tom prípade iteruje normálne, bez obmedzení, nejaké $a^s, ~ s<\vert \alpha \vert \leq N < \vert w \vert$. Potom ak $w=a^t$, tak $x=a^{t-s},y=a^s, k=1$ a $xy^j \in L(\alpha) ~ \forall j$. 
\item Lookaround zasahujúci do iterujúcej $*$ nie je ohraničený, t.j. ak je to lookahead, neobsahuje na konci $\mathdollar$, a ak je to lookbehind, neobsahuje na začiatku $\textasciicircum$. Potom ho podľa vety \ref{bezprefixove} vieme prepísať tak, že bude matchovať len najkratšie slovo z jeho jazyka. Získame tak konečný jazyk prefixov/sufixov, ktorý síce ovplyvňuje dĺžku matchovaných slov, ale iba tú minimálnu. Teda $*$ iteruje bez obmedzení, čo je predošlý prípad.
\item Našli sme iterujúcu $*$ a zasahuje do nej lookaround matchujúci až po koniec resp. začiatok slova. Buď je to lookbehind (nachádzajúci sa za $*$), za ktorým už v $\alpha$ nie je žiadne iterovanie alebo je to lookahead (nachádzajúci sa pred $*$), prípadne oba alebo ich môže byť viac. Každopádne, keďže je slovo dostatočne dlhé, aj samotný lookaround musí niečo pumpovať. Vyriešme to pre prípad jedného lookaroundu a potom uvidíme, že tie ostatné z toho vyplývajú.

Máme $w=a^e$. V lookarounde sa pumpuje nejaké $a^f$, v $*$ mimo neho $a^g$ (ak je na výber viac možností, stačí vybrať jednu, napr. tú najkratšiu). Potom lookaround matchuje $a^{e+h*f}$ a $*$ mimo neho $a^{e+i*g}$, pre ľubovoľné $h,i \in \N_0$. Nech $b = nsn(f,g)$, prienik týchto slov je $a^{e+b*j} \in L(\alpha), ~ \forall j \in \N_0$. Tvrdíme $b<e$. A naozaj, keďže $a^e \in L(\alpha)$ a zároveň lookaround aj $*$ určite niečo pumpovali, ich najmenšia možná zhoda je práve $b$.

Pracujeme s le-regexami, takže sa niekde musia objaviť aj spätné referencie\footnote{Keby sa v $\alpha$ nenachádzali, podľa vety \ref{lookahead+R} máme regulárny jazyk, pre ktorý pumpovacia lema existuje.}. Uvedomme si, že lookaroundy sa môžu vyskytovať niekde medzi nimi, takže keď napríklad $*$ pumpuje $a^m$, celkovo sa v slove pumpuje $a^{n*m}$, kde $n$ je počet referencií ukazujúcich na túto $*$. Tým, že sa lookaroundy budú nachádzať medzi nimi, môže nastať, že pre jeden sa ďalej v slove pumpuje $a^{n_1*m}$ a pre druhý $a^{n_2*m}$, kde $n_1 \neq n_2$. Ak $*$ vie pumpovať $a^m$, tak vie aj $a^{n!*m}$, čo by vyriešilo všetky možné násobky, ktoré ovplyvňujú rôzne lookaroundy. Podľa toho $k = n!$, pre $n$ ako najvyšší počet spätných referencií na nejakú iteráciu, ktorú využívame.

Celkovo máme $x=a^{e-b},~y=a^b$ a $k = n!$ ak je určené, inak $k=1$. Z toho $xy^{kj} = a^{e-b}a^{bkj} = a^{e+bj'} \in L(\alpha) ~ \forall j \in \N, ~ j' = kj-1$. 

V prípade viacerých lookaroundov riešime postupne -- vždy vyrátané $b$ bude v ďalšom kole novým $g$. Podobne riešime vnorené lookaroundy -- postupujeme zvnútra von (musí ich byť konečne veľa a ten najvnútornejší z nich isto bude obsahovať už len Eregex).
\end{enumerate} 
\end{proof}

\paragraph{Poznámka 1.}
Spomínaný prípad, pre ktorý je problematické túto vetu dokázať má problém aj s definovaním dostatočne veľkého slova. Pozrime sa napríklad na le-regex $\beta = ((?=(a^m)*\mathdollar )a^{m+1})*a \lbrace 1,m-1 \rbrace \mathdollar$, aké slová obsahuje?
\begin{itemize}
\item $a^{m+1+(m-1)}$
\item $a^{2m+2+(m-2)}$ \\
	  ~~~~~~~\vdots
\item $a^{(m-1)(m+1)+1}$
\end{itemize}
avšak $a^{m(m+1)} \notin L(\beta )$, lebo nám chýba zvyšok $0$. Teda vidíme, že le-regex využíva iteráciu $(m-1)$--krát a napriek tomu je konečný.

\paragraph{Poznámka 2.}
Zoberme si triedu jazykov nad takýmito le-regexami. Veta nám hovorí čosi o zložitosti jej jazykov. Napriklad $L = \lbrace a^m ~|~ m~je~prvočíslo\rbrace$ do nej nepatrí, lebo sa nedá vôbec pumpovať -- medzi prvočíslami nenájdeme žiadne násobky. Z toho už vyplýva, že je vlastnou podmnožinou $\L_{CS}$ a nie je uzavretá na komplement ($L^c = (aaa*)\backslash 1(\backslash1)* \in Eregex$).

\paragraph{Poznámka 3.}
V prípade, že by sa dokázala pumpovacia lema, vedeli by sme ukázať, že LEregex nie je uzavretá na vymazávajúci homomorfizmus. Príkladom je jazyk $L =\lbrace (a^m\#a^{m(m+1)k}\#)* ~|~ k,m \in \N;~ m ~ sa ~ priebežne~mení \rbrace$ popísaný regexom $\alpha = ((a*)\#(?=\backslash 2*\#)(\backslash 2a)*\#)*$. Vymazávací homomorfizmus $h$, kde $h(a) = a, ~ h(\#) = \lbrace  \varepsilon$ by zrušil mriežky, teda $h(L) = a^{\sum_{i=0}^n m_i(m_i+1)k_i}~|~ n,m_i,k_i \in \N ~ \forall i \rbrace$. Riešnením nemôže byť $h(\alpha)$, lebo bez mriežok je v lookaheade prefixový jazyk, teda matchuje najkratší prefix a nekontroluje deliteľnosť $m_i$. \todo

\section{Negatívny lookaround}\label{chap:negla}

Pripomeňme si, že le-regexy obohatené o negatívny lookaround sme nazvali nle-regexy a triedu nad nimi nLEregex.

\begin{veta}
Trieda nLEregex je uzavretá na prienik.
\end{veta}
\begin{proof}
Podobne ako veta \ref{prienik}.
\end{proof}

\begin{veta}
Trieda nLEregex je uzavretá na komplement.
\end{veta}
\begin{proof}
Nech $L_1 \in nLEregex$, potom $L_1^c  = \left( ?! L_1 \mathdollar \right) .* \mathdollar $.
\end{proof}

\paragraph{Poznámka.} Negatívne verzie lookaroundu sa v podstate veľmi nelíšia od ich tých pozitívnych. Robia to isté, akurát zmenia výslednú odpoveď. Preto nečakáme nové zložité kolízie s operáciami.

\todo{trošku charakterizuj triedu} \todo{pozri sa na záver...}
