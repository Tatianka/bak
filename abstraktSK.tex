$\\$
Objektom skúmania našej práce boli nové konštrukcie regulárnych výrazov, ktoré vznikli vývojom programovacích jazykov. Zamerali sme sa na operácie lookahead a lookbehind zvlášť v ich pozitívnej a negatívnej forme. Skúmali sme uzavretosť tried Chomského hierarchie na tieto operácie. Ďalej sme skúmali lookaround skombinovaný s rôznymi množinami operácií regulárnych výrazov. Zistili sme, že základná množina s lookaroundom pokryla iba triedu regulárnych jazykov. V množine rozšírenej o spätné referencie a pozitívny lookaround sme vytvorili nové zložitejšie jazyky, pre ktoré nebolo možné dokázať pumpovaciu lemu. Pridanie lookaroundu prinieslo uzavretosť na prienik a jeho negatívna verzia uzavretosť na komplement.
