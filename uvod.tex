%\cleardoublepage
\phantomsection
\addcontentsline{toc}{chapter}{Úvod}
\chapter*{Úvod}\label{chap:intro}

Regulárne výrazy vznikli ako ďalší teoretický model popisujúci triedu regulárnych jazykov. Pre jednoduchý zápis sa ho medzi prvými ujal Ken Thomson a implementoval ho do textového editora. Použitím prevodu na NFA mu zaručil lineárnu časovú zložitosť. Po rokoch a ďalších efektívnych implementáciách však tieto algoritmy ušli pozornosti Hernyho Spencera pri reimplementovaní a zverejnení ôsmej verzie Unixu, kam pre regulárne výrazy napísal algoritmus s exponenciálnou časovou zložitosťou využívajúci backtraking. Tento oveľa pomalší algortimus sa neskôr stal základom pre programovacie jazyky ako Perl, PCRE, Python, atď. \cite{Cox07SlowPython}

S rozvojom jazykov sa vyvíjali aj regulárne výrazy a získavali oproti teoretickému modelu nové konštrukcie. Mnohé z nich majú dlhší ekvivalent používajúci tradičnú syntax. Na druhej strane vytŕčajú napríklad spätné referencie, ktorými sme schopný popísať až kontextový jazyk.

Nás budú zaujímať práve tieto nové konštrukcie, primárne lookahead a lookbehind. Spomínané spätné referencie už boli skúmané, teda uvedieme dosiahnuté výsledky a budeme s nimi pracovať. Bude nás zaujímať, akú triedu jazykov model s pridanou novou konštrukciou popisuje v rámci Chomského hierarchie a aké uzáverové vlastnosti sa podarilo dosiahnuť.

Skúmaním nových konštrukcií z teoretického hľadiska umožníme programátorom lepší náhľad na to, čo skutočne dokážu a aký zložitý algoritmus bude treba použiť na ich implementáciu. Trieda regulárnych jazykov vystačí s ľahko naprogramovateľnými konečnými automatmi, avšak vyššie triedy vyžadujú backtracking, ktorý samozrejme znamená väčšiu časovú zložitosť.
