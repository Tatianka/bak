%\cleardoublepage
\phantomsection
\addcontentsline{toc}{chapter}{Úvod}
\chapter*{Úvod}\label{chap:intro}

Bla bla úvodné info o regulárnych výrazoch:
- vzniklo ako teória
- niekto implementoval do textových editorov, neskôr Thomson do Unixu
- odtiaľ sa rozšírili do programovacích jazykov
- programovacie jazyky rozširujú svoju funkcionalitu, to platí aj pre časť regulárnych výrazov. Rozšírením o nové konštrukcie už nemusia patriť do triedy regulárnych jazykov. Mnohé konštrukcie sú len kozmetické úpravy a pomôcky, ktoré nezosilnia daný model. Zaujímavé sú tie, ktoré už pomôžu vytvoriť (akceptovať) jazyky z vyšších tried Chomského hierarchie.
- Zaradením triedy jazykov aktuálneho modelu do Chomského hierarchie môže dopomôcť pri implementácii jednotlivých operácií. Trieda regulárnych jazykov vystačí s ľahko naprogramovateľnými konečnými automatmi, avšak vyššie triedy vyžadujú backtracking, ktorý samozrejme znamená väčšiu časovú zložitosť.
