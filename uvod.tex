%\cleardoublepage
\phantomsection
\addcontentsline{toc}{chapter}{Úvod}
\chapter*{Úvod}\label{chap:intro}

Regulárne výrazy vznikli ako ďalší teoretický model popisujúci triedu regulárnych jazykov. Pre jednoduchý zápis sa ho v sedemdesiatych rokoch 20. storočia medzi prvými ujal Ken Thomson a implementoval ho do textového editora. Myšlienka sa uchytila a rozšírila do Unixu -- editor ed a filter grep -- a odtiaľ neskôr do programovacích jazykov ako Perl, Python, Ruby, atď. \cite{Cox07SlowPython}

S rozvojom jazykov sa vyvíjali aj regulárne výrazy a získavali oproti teoretickému modelu nové konštrukcie. Mnohé z nich majú dlhší ekvivalent používajúci tradičnú syntax. Na druhej strane vytŕčajú napríklad spätné referencie, ktorými sme schopní popísať až kontextový jazyk.

Nás budú zaujímať práve tieto nové konštrukcie, primárne lookahead a lookbehind. Spomínané spätné referencie už boli skúmané, teda uvedieme dosiahnuté výsledky a budeme s nimi pracovať. Bude nás zaujímať, akú triedu jazykov model s pridanou novou konštrukciou popisuje v rámci Chomského hierarchie a aké uzáverové vlastnosti sa podarilo dosiahnuť.

Skúmaním nových konštrukcií z teoretického hľadiska umožníme programátorom lepší náhľad na to, čo skutočne dokážu a aký zložitý algoritmus bude treba použiť na ich implementáciu. Trieda regulárnych jazykov vystačí s ľahko naprogramovateľnými konečnými automatmi, avšak vyššie triedy vyžadujú backtracking, ktorý samozrejme znamená väčšiu časovú zložitosť.
