%\cleardoublepage
\phantomsection
\addcontentsline{toc}{chapter}{Úvod}
\chapter*{Úvod}\label{chap:intro}

Bla bla úvodné info o regulárnych výrazoch:
- vzniklo ako teória
- niekto implementoval do textových editorov, neskôr Thomson do Unixu
- odtiaľ sa rozšírili do programovacích jazykov
- programovacie jazyky rozširujú svoju funkcionalitu, to platí aj pre časť regulárnych výrazov. Rozšírením o nové konštrukcie už nemusia patriť do triedy regulárnych jazykov. Mnohé konštrukcie sú len kozmetické úpravy a pomôcky, ktoré nezosilnia daný model. Zaujímavé sú tie, ktoré už pomôžu vytvoriť (akceptovať) jazyky z vyšších tried Chomského hierarchie.
- Zaradením triedy jazykov aktuálneho modelu do Chomského hierarchie môže dopomôcť pri implementácii jednotlivých operácií. Trieda regulárnych jazykov vystačí s ľahko naprogramovateľnými konečnými automatmi, avšak vyššie triedy vyžadujú backtracking, ktorý samozrejme znamená väčšiu časovú zložitosť.

\section{Základná forma regulárnych výrazov}\label{chap:zaklad}

Keďže implementované regulárne výrazy sa už natoľko líšia od počiatočného teoretického modelu, zaužíval sa pre ne názov \textbf{regexy}. Budem ho používať aj ja a prípadnými predponami budem rozlišovať, ktorú množinu operácií práve myslím. 

Pojem \textbf{regex} bude slúžiť na pomenovanie regulárnych výrazov, ktoré pokrývajú triedu regulárnych jazykov. Pre ozrejmenie uvediem základnú definíciu regexu z článku \cite{ExtendedRegexPower}. Niektoré konštrukcie sú oproti teoretickému modelu nové, ale dôkaz toho, že pokrýva stále rovnakú triedu jazykov, je triviálny.

\underline{Základná forma regexov}
\begin{list}{}{}
\item[(1)] Pre každé $a \in \Sigma$, $a$ je regex a $L(a)=\lbrace a \rbrace$. Poznamenajme, že pre každé $x \in \lbrace (,), \{, \},[,],\mathdollar,|, \backslash, .,?,*,+ \rbrace, \backslash x \in \Sigma $ a je regexom a $L(\backslash x) = \lbrace x \rbrace$. Naviac aj $\backslash n$ a $\backslash t$ patria do $\Sigma$ a oba sú regexami. $L(\backslash n)$ a $L(\backslash t)$ popisujú jazyky skladajúce sa z nového riadku a tabulátora.
\item[(2)] Pre regexy $e_1$ a $e_2$ 
\begin{list}{}{}
\item $(e_1)(e_2)$ (zreťazenie), 
\item $(e_1)|(e_2)$ (alternácia), a 
\item $(e_1)*$ (Kleeneho uzáver) 
\end{list}
sú regexy, kde $L((e_1)(e_2)) = L(e_1)L(e_2), L((e_1)|(e_2))=L(e_1) \cup L(e_2)$ a $L((e_1)*) = (L(e_1))^*$. Okrúhle zátvorky môžu byť vynechané. Ak sú vynechané, alternácia, zreťazenie a Kleeneho uzáver majú vyššiu prioritu.
\item[(3)] Regex je tvorený konečným počtom prvkov z (1) a (2).
\end{list}

\underline{Skrátnená forma}
\begin{list}{}{}
\item[(1)] Pre každý regex $e$: $(e)+$ je regex a $(e)+ \equiv e(e)*$.
\item[(2)] Znak ' . ' znamená ľubovolný znak okrem $\backslash n$.
\end{list}

\underline{Triedy znakov}
\begin{list}{}{}
\item[(1)] Pre $a_{i_1},a_{i_2},\dots ,a_{i_t} \in \Sigma,~t \geq 1,~\left[ a_{i_1}a_{i_2}\dots a_{i_t} \right] \equiv a_{i_1}|a_{i_2}|\dots |a_{i_t} $.
\item[(2)] Pre $a_i,a_j \in \Sigma$ také, že $a_i\leq a_j,~ [a_i-a_j]$ je regex a $[a_i-a_j]\equiv a_i|a_{i+1}|\dots |a_j$.
\item[(3)] Pre $a_{i_1},a_{i_2},\dots ,a_{i_t} \in \Sigma,~t \geq 1,~\left[ \textasciicircum a_{i_1}a_{i_2}\dots a_{i_t} \right] \equiv b_{i_1}|b_{i_2}|\dots |b_{i_s} $, kde $\lbrace b_{i_1}|b_{i_2}|\dots |b_{i_s}\rbrace = \Sigma - \lbrace a_{i_1},a_{i_2},\dots ,a_{i_t} \rbrace$.
\item[(4)] Pre $a_i,a_j \in \Sigma$ také, že $a_i\leq a_j,~ [a_i-a_j]$ je regex a $[\textasciicircum a_i-a_j]\equiv b_{i_1}|b_{i_2}|\dots |b_{i_s}$, kde $\lbrace b_{i_1}|b_{i_2}|\dots |b_{i_s}\rbrace = \Sigma - \lbrace a_i|a_{i+1}|\dots |a_j \rbrace$.
\item[(5)] Zmes (1) a (2) alebo (3) a (4).
\end{list}

\underline{Ukotvenie}
\begin{list}{}{}
\item[(1)] Znak pre začiatok riadku $ \textasciicircum $.
\item[(2)] Znak pre koniec riadku $ \mathdollar $.
\end{list}